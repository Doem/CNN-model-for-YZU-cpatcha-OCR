%Copyright 2014 Jean-Philippe Eisenbarth
%This program is free software: you can 
%redistribute it and/or modify it under the terms of the GNU General Public 
%License as published by the Free Software Foundation, either version 3 of the 
%License, or (at your option) any later version.
%This program is distributed in the hope that it will be useful,but WITHOUT ANY 
%WARRANTY; without even the implied warranty of MERCHANTABILITY or FITNESS FOR A 
%PARTICULAR PURPOSE. See the GNU General Public License for more details.
%You should have received a copy of the GNU General Public License along with 
%this program.  If not, see <http://www.gnu.org/licenses/>.

%Based on the code of Yiannis Lazarides
%http://tex.stackexchange.com/questions/42602/software-requirements-specification-with-latex
%http://tex.stackexchange.com/users/963/yiannis-lazarides
%Also based on the template of Karl E. Wiegers
%http://www.se.rit.edu/~emad/teaching/slides/srs_template_sep14.pdf
%http://karlwiegers.com
\documentclass{scrreprt}
\usepackage{multirow}
\usepackage{graphicx}
\usepackage[table]{xcolor}
\usepackage{listings}
\usepackage{underscore}
\usepackage[bookmarks=true]{hyperref}
\usepackage[utf8]{inputenc}
\usepackage[english]{babel}
\usepackage{CJKutf8}
\usepackage{indentfirst}
\hypersetup{
    bookmarks=false,    % show bookmarks bar?
    pdftitle={Software Requirement Specification},    % title
    pdfauthor={Jean-Philippe Eisenbarth},                     % author
    pdfsubject={TeX and LaTeX},                        % subject of the document
    pdfkeywords={TeX, LaTeX, graphics, images}, % list of keywords
    colorlinks=true,       % false: boxed links; true: colored links
    linkcolor=blue,       % color of internal links
    citecolor=black,       % color of links to bibliography
    filecolor=black,        % color of file links
    urlcolor=purple,        % color of external links
    linktoc=page            % only page is linked
}%
\def\myversion{1.0 }
\date{}
%\title
\usepackage{hyperref}
\begin{document}
\begin{CJK}{UTF8}{nkai}

\begin{flushright}
    \begin{bfseries}
        \Huge{基於CNN之元智選課系統驗證碼辨識}\\

        \vspace{8cm}
       \vspace{0.2cm}
       Teammate : 1051522 許致瑋\\
       \vspace{0.2cm}
       1051523 石曜愷\\
\vspace{0.2cm}
       1051528 官政凱\\
\vspace{0.2cm}
       1053327 陳遠安\\
\vspace{0.2cm}
       1053329 陳明彥\\

        \vspace{1cm}
        \today\\
    \end{bfseries}
\end{flushright}

\tableofcontents


\chapter{Introduction}

\section{Purpose}

\setlength{\parindent}{1.5em}此規格書目的在於概述本專案之軟體需求,其中包含專案目的、實作、目標用戶與操作資訊。\\
\setlength{\parindent}{1.5em}本專案利用CNN方式訓練神經網路模型來辨識元智選課系統的驗證碼,以此可供其他開發人員開發對於選課系統之自動化相關專案。
\section{Intended Audience and Reading Suggestions}

此規格書之閱讀對象:
\begin{itemize}
  \item  專案經理: 了解專案預期之功能,分析並檢視是否符合需求
  \item 開發者: 了解專案架構與功能需求,加以分析並開發出符合的規格的專案
  \item 使用者: 了解專案的架構與使用方式,運用在其他相關專案中。
\end{itemize}
          
\section{Project Scope}
\setlength{\parindent}{1.5em}本專案除了CNN模型之外,還包含了訓練時所用的訓練集、收集訓練集所用的爬蟲程式以及運用CNN模型來自動登入選課系統之範例程式。

\chapter{Overall Description}

\section{Product Perspective}

\begin{figure}[h]
\begin{center}
\includegraphics[width=8.5cm]{456.png}
\end{center}
\caption{架構圖}
\label{fig:1}
\end{figure}

\section{Product Functions}

\begin{itemize}
  \item  登入選課系統
  \item 抓網頁的驗證碼圖片
  \item .辨識驗證碼
\end{itemize}

\section{User Classes and Characteristics}
我們產品的主要客戶群就是?

\section{Operating Environment}
browser必須是chrome。
\section{Design and Implementation Constraints}
\setlength{\parindent}{1.5em}資料蒐集: ?
\section{Assumptions and Dependencies}
\setlength{\parindent}{1.5em}驗證碼格式要類似。

\chapter{External Interface Requirements}

\section{User Interfaces}
\setlength{\parindent}{1.5em}打開terminal,打python xxx.py
Test.
\begin{figure}[h]
\begin{center}
\includegraphics[width=8.5cm]{123.png}
\end{center}
\caption{編譯畫面}
\label{fig:1}
\end{figure}
\section{Hardware Interfaces}
沒
\section{Software Interfaces}
\begin{enumerate}
  \item  安裝軟體:\\
         keras、tensorflow、python、opencv、selenium-wire、bs4、pytesseract、requests、configparser
  \item  使用的參數:\\
loss function: categorical crossentropy\\
       optimizer: Adam(預設參數)\\
       metrics: accuracy\\
  \item  輸入與輸出:\\
 輸入 :驗證碼圖片\\
       輸出 :驗證碼的答案\\
  \item  輸入與輸出大小:\\
 輸入大小 :20*60\\
       輸出大小 :四個字\\
\end{enumerate}


\chapter{System Features}

\section{Description and Priority}
\setlength{\parindent}{1.5em}系統會自己打開chrome,自動登入選課系統。
\section{Stimulus/Response Sequences}
\begin{tabular}{ | l | l | p{5cm} |}
\hline
case1 & 登入成功:停在登入以後的畫面。\\ \hline
 case2 & 驗證碼辦認錯誤:跳出警告,然後重新整理再試一次。\\ \hline
 case3 & 帳密打錯:跳出打錯帳號密碼訊息,並且直接結束程式。\\ \hline
 case4 & 被學校擋:跳出錯誤訊息,並且直接結束程式。\\
\hline
\end{tabular}

\section{Functional Requirements}
\begin{itemize}
  \item  辨識驗證碼
  \item 自動登入
\end{itemize}

\chapter{Other Nonfunctional Requirements}

\section{Performance Requirements}
\setlength{\parindent}{1.5em}準確率:驗證碼辨識率至少高於90\%。
\section{Safety Requirements (optional)}
無

\section{Security Requirements (optional)}
無



\end{CJK}

\end{document}
